% Options for packages loaded elsewhere
\PassOptionsToPackage{unicode}{hyperref}
\PassOptionsToPackage{hyphens}{url}
%
\documentclass[
]{article}
\usepackage{amsmath,amssymb}
\usepackage{lmodern}
\usepackage{iftex}
\ifPDFTeX
  \usepackage[T1]{fontenc}
  \usepackage[utf8]{inputenc}
  \usepackage{textcomp} % provide euro and other symbols
\else % if luatex or xetex
  \usepackage{unicode-math}
  \defaultfontfeatures{Scale=MatchLowercase}
  \defaultfontfeatures[\rmfamily]{Ligatures=TeX,Scale=1}
\fi
% Use upquote if available, for straight quotes in verbatim environments
\IfFileExists{upquote.sty}{\usepackage{upquote}}{}
\IfFileExists{microtype.sty}{% use microtype if available
  \usepackage[]{microtype}
  \UseMicrotypeSet[protrusion]{basicmath} % disable protrusion for tt fonts
}{}
\makeatletter
\@ifundefined{KOMAClassName}{% if non-KOMA class
  \IfFileExists{parskip.sty}{%
    \usepackage{parskip}
  }{% else
    \setlength{\parindent}{0pt}
    \setlength{\parskip}{6pt plus 2pt minus 1pt}}
}{% if KOMA class
  \KOMAoptions{parskip=half}}
\makeatother
\usepackage{xcolor}
\IfFileExists{xurl.sty}{\usepackage{xurl}}{} % add URL line breaks if available
\IfFileExists{bookmark.sty}{\usepackage{bookmark}}{\usepackage{hyperref}}
\hypersetup{
  pdftitle={Survey Data Analysis report},
  pdfauthor={SHUANG SONG},
  hidelinks,
  pdfcreator={LaTeX via pandoc}}
\urlstyle{same} % disable monospaced font for URLs
\usepackage[margin=1in]{geometry}
\usepackage{color}
\usepackage{fancyvrb}
\newcommand{\VerbBar}{|}
\newcommand{\VERB}{\Verb[commandchars=\\\{\}]}
\DefineVerbatimEnvironment{Highlighting}{Verbatim}{commandchars=\\\{\}}
% Add ',fontsize=\small' for more characters per line
\usepackage{framed}
\definecolor{shadecolor}{RGB}{248,248,248}
\newenvironment{Shaded}{\begin{snugshade}}{\end{snugshade}}
\newcommand{\AlertTok}[1]{\textcolor[rgb]{0.94,0.16,0.16}{#1}}
\newcommand{\AnnotationTok}[1]{\textcolor[rgb]{0.56,0.35,0.01}{\textbf{\textit{#1}}}}
\newcommand{\AttributeTok}[1]{\textcolor[rgb]{0.77,0.63,0.00}{#1}}
\newcommand{\BaseNTok}[1]{\textcolor[rgb]{0.00,0.00,0.81}{#1}}
\newcommand{\BuiltInTok}[1]{#1}
\newcommand{\CharTok}[1]{\textcolor[rgb]{0.31,0.60,0.02}{#1}}
\newcommand{\CommentTok}[1]{\textcolor[rgb]{0.56,0.35,0.01}{\textit{#1}}}
\newcommand{\CommentVarTok}[1]{\textcolor[rgb]{0.56,0.35,0.01}{\textbf{\textit{#1}}}}
\newcommand{\ConstantTok}[1]{\textcolor[rgb]{0.00,0.00,0.00}{#1}}
\newcommand{\ControlFlowTok}[1]{\textcolor[rgb]{0.13,0.29,0.53}{\textbf{#1}}}
\newcommand{\DataTypeTok}[1]{\textcolor[rgb]{0.13,0.29,0.53}{#1}}
\newcommand{\DecValTok}[1]{\textcolor[rgb]{0.00,0.00,0.81}{#1}}
\newcommand{\DocumentationTok}[1]{\textcolor[rgb]{0.56,0.35,0.01}{\textbf{\textit{#1}}}}
\newcommand{\ErrorTok}[1]{\textcolor[rgb]{0.64,0.00,0.00}{\textbf{#1}}}
\newcommand{\ExtensionTok}[1]{#1}
\newcommand{\FloatTok}[1]{\textcolor[rgb]{0.00,0.00,0.81}{#1}}
\newcommand{\FunctionTok}[1]{\textcolor[rgb]{0.00,0.00,0.00}{#1}}
\newcommand{\ImportTok}[1]{#1}
\newcommand{\InformationTok}[1]{\textcolor[rgb]{0.56,0.35,0.01}{\textbf{\textit{#1}}}}
\newcommand{\KeywordTok}[1]{\textcolor[rgb]{0.13,0.29,0.53}{\textbf{#1}}}
\newcommand{\NormalTok}[1]{#1}
\newcommand{\OperatorTok}[1]{\textcolor[rgb]{0.81,0.36,0.00}{\textbf{#1}}}
\newcommand{\OtherTok}[1]{\textcolor[rgb]{0.56,0.35,0.01}{#1}}
\newcommand{\PreprocessorTok}[1]{\textcolor[rgb]{0.56,0.35,0.01}{\textit{#1}}}
\newcommand{\RegionMarkerTok}[1]{#1}
\newcommand{\SpecialCharTok}[1]{\textcolor[rgb]{0.00,0.00,0.00}{#1}}
\newcommand{\SpecialStringTok}[1]{\textcolor[rgb]{0.31,0.60,0.02}{#1}}
\newcommand{\StringTok}[1]{\textcolor[rgb]{0.31,0.60,0.02}{#1}}
\newcommand{\VariableTok}[1]{\textcolor[rgb]{0.00,0.00,0.00}{#1}}
\newcommand{\VerbatimStringTok}[1]{\textcolor[rgb]{0.31,0.60,0.02}{#1}}
\newcommand{\WarningTok}[1]{\textcolor[rgb]{0.56,0.35,0.01}{\textbf{\textit{#1}}}}
\usepackage{graphicx}
\makeatletter
\def\maxwidth{\ifdim\Gin@nat@width>\linewidth\linewidth\else\Gin@nat@width\fi}
\def\maxheight{\ifdim\Gin@nat@height>\textheight\textheight\else\Gin@nat@height\fi}
\makeatother
% Scale images if necessary, so that they will not overflow the page
% margins by default, and it is still possible to overwrite the defaults
% using explicit options in \includegraphics[width, height, ...]{}
\setkeys{Gin}{width=\maxwidth,height=\maxheight,keepaspectratio}
% Set default figure placement to htbp
\makeatletter
\def\fps@figure{htbp}
\makeatother
\setlength{\emergencystretch}{3em} % prevent overfull lines
\providecommand{\tightlist}{%
  \setlength{\itemsep}{0pt}\setlength{\parskip}{0pt}}
\setcounter{secnumdepth}{-\maxdimen} % remove section numbering
\ifLuaTeX
  \usepackage{selnolig}  % disable illegal ligatures
\fi

\title{Survey Data Analysis report}
\author{SHUANG SONG}
\date{2022-03-02}

\begin{document}
\maketitle

\hypertarget{how-did-you-prepare-the-data-for-analysis}{%
\subsubsection{1. How did you prepare the data for
analysis?}\label{how-did-you-prepare-the-data-for-analysis}}

\begin{itemize}
\tightlist
\item
  First I loaded the data using R language, by exploring the data, there
  are 668 rows and 125 columns.
\item
  Second, reformat the column names. In the raw data there are question
  mark, dash, space, dot that is not suitable and standard for column
  names, so I filter them out and replaced them with either underscore,
  or no space.
\item
  From instruction on powerpoint, we need to exclude the international
  student, so I filtered and excluded them from the dataframe.
\item
  Now the data is much more clean and ready for further analysis. See
  code below.
\end{itemize}

load the packages we need:

\begin{Shaded}
\begin{Highlighting}[]
\FunctionTok{rm}\NormalTok{(}\AttributeTok{list =} \FunctionTok{ls}\NormalTok{()) }\CommentTok{\# clear the environment}
\FunctionTok{library}\NormalTok{(readxl)}
\FunctionTok{library}\NormalTok{(writexl)}
\FunctionTok{library}\NormalTok{(dplyr)}
\end{Highlighting}
\end{Shaded}

\begin{verbatim}
## 
## Attaching package: 'dplyr'
\end{verbatim}

\begin{verbatim}
## The following objects are masked from 'package:stats':
## 
##     filter, lag
\end{verbatim}

\begin{verbatim}
## The following objects are masked from 'package:base':
## 
##     intersect, setdiff, setequal, union
\end{verbatim}

\begin{Shaded}
\begin{Highlighting}[]
\FunctionTok{library}\NormalTok{(tidyr)}
\FunctionTok{library}\NormalTok{(ggplot2)}
\FunctionTok{library}\NormalTok{(scales)}
\FunctionTok{library}\NormalTok{(ggpubr)}
\end{Highlighting}
\end{Shaded}

read in xlsx file :

\begin{Shaded}
\begin{Highlighting}[]
\NormalTok{file}\OtherTok{\textless{}{-}} \FunctionTok{read\_excel}\NormalTok{(}\StringTok{"/Users/cleopathy/Desktop/assignment1.xlsx"}\NormalTok{, }\AttributeTok{sheet =} \StringTok{"RAW DATA {-} DEIDENTIFIED"}\NormalTok{)}
\end{Highlighting}
\end{Shaded}

\begin{verbatim}
## New names:
## * respect -> respect...9
## * intclimate -> intclimate...10
## * socclimate -> socclimate...11
## * collegial -> collegial...12
## * sameuniv -> sameuniv...13
## * ...
\end{verbatim}

excluded international student:

\begin{Shaded}
\begin{Highlighting}[]
\NormalTok{file }\OtherTok{\textless{}{-}} \FunctionTok{filter}\NormalTok{(file, Q62 }\SpecialCharTok{!=} \StringTok{"International (Non{-}U.S. Citizen with temporary U.S. Visa)"}\NormalTok{)}
\end{Highlighting}
\end{Shaded}

reformat the column names:

\begin{Shaded}
\begin{Highlighting}[]
\CommentTok{\#reformat column name}
\FunctionTok{colnames}\NormalTok{(file) }\OtherTok{\textless{}{-}} \FunctionTok{gsub}\NormalTok{(}\StringTok{"/"}\NormalTok{, }\StringTok{"\_"}\NormalTok{, }\FunctionTok{colnames}\NormalTok{(file))}
\FunctionTok{colnames}\NormalTok{(file) }\OtherTok{\textless{}{-}} \FunctionTok{gsub}\NormalTok{(}\StringTok{" "}\NormalTok{, }\StringTok{"\_"}\NormalTok{, }\FunctionTok{colnames}\NormalTok{(file))}
\FunctionTok{colnames}\NormalTok{(file) }\OtherTok{\textless{}{-}} \FunctionTok{gsub}\NormalTok{(}\StringTok{"}\SpecialCharTok{\textbackslash{}\textbackslash{}}\StringTok{?"}\NormalTok{, }\StringTok{""}\NormalTok{, }\FunctionTok{colnames}\NormalTok{(file))}
\FunctionTok{colnames}\NormalTok{(file) }\OtherTok{\textless{}{-}} \FunctionTok{gsub}\NormalTok{(}\StringTok{"}\SpecialCharTok{\textbackslash{}\textbackslash{}}\StringTok{."}\NormalTok{, }\StringTok{""}\NormalTok{, }\FunctionTok{colnames}\NormalTok{(file))}
\FunctionTok{head}\NormalTok{(file)}
\end{Highlighting}
\end{Shaded}

\begin{verbatim}
## # A tibble: 6 x 125
##   ResponseId        Q6    Q8    `STEM_Non-STEM` Q12   satacad satlife satoverall
##   <chr>             <chr> <chr> <chr>           <chr> <chr>   <chr>   <chr>     
## 1 R_2cCN179YVYmv5Cl PhD   Coll~ Non-STEM        <NA>  Very g~ Good    Good      
## 2 R_6QjQtKUFNbhLsoV PhD   Coll~ Non-STEM        <NA>  Excell~ Good    Excellent 
## 3 R_UEkRKRes2mmR5bX PhD   Coll~ STEM            <NA>  Very g~ Good    Good      
## 4 R_3LgLwQCm1Il114u PhD   Coll~ Non-STEM        I wi~ Very g~ Good    Very good 
## 5 R_3VmQbgty78CG8Df PhD   Coll~ STEM            I wi~ Very g~ Very g~ Very good 
## 6 R_3QMoPUV7Zwm16na PhD   Coll~ STEM            I wi~ Fair    Fair    Fair      
## # ... with 117 more variables: respect9 <chr>, intclimate10 <chr>,
## #   socclimate11 <chr>, collegial12 <chr>, sameuniv13 <chr>, samefield14 <chr>,
## #   recommend15 <chr>, curriculum16 <chr>, teaching17 <chr>, advising18 <chr>,
## #   candidacy19 <chr>, interdisc20 <chr>, employment21 <chr>, progqual22 <chr>,
## #   Q21 <chr>, Q22 <chr>, Q23 <chr>, Q24 <chr>, finance <chr>, infotech <chr>,
## #   space <chr>, library <chr>, lab <chr>, htopicadv <chr>, hresearchadv <chr>,
## #   hwritingadv <chr>, hacadadv <chr>, hnonacadadv <chr>, hemployadv <chr>, ...
\end{verbatim}

\hypertarget{how-did-you-address-any-missing-values}{%
\subsubsection{2. How did you address any missing
values?}\label{how-did-you-address-any-missing-values}}

First, I did analysis and find out if there are cases that are completed
empty (full of NAs), these records should be excluded for the further
analysis.

Second, find out how many NAs in each row and each column, if the NA
rate is higher than a cut out rate, for example, 50\%, then the row and
columns should be excluded.

For rows and columns that has lower NA rate, we need to remove NAs and
replaced them with a value. Currently, we need imputation. Imputation is
especially important in advanced data analysis. There are lots of
methods of data imputation, for this analysis, I used averaged
imputation. In this way, missing values are taken care of.

\hypertarget{which-statistical-methods-did-you-use-for-your-data-analysis-and-why}{%
\subsubsection{3. Which statistical methods did you use for your data
analysis, and
why?}\label{which-statistical-methods-did-you-use-for-your-data-analysis-and-why}}

When compare if there is a significant difference between URG students'
group and non-URG students' group for the categories in question faculty
mentoring and advising, I used pairwise t-test to calculate p value. In
those categories: selection of a dissertation topic, your dissertation
research, writing and revising your dissertation, academic career
option, nonacademic career option, search for employment or training, I
found that there is no significant statistical difference between URG
and Non-URG students as p value is larger than 0.05.

I also convert likert scale data to numbers and used means of score to
visualize the program quality score bar plot.

In the category of selection of a dissertation topic, I used ANOVA to
conduct an analysis showing that there is no significant difference of
extent of helpful on this category between different ethnicity/race
group.

\hypertarget{did-you-determine-response-rates-for-the-different-questions-why-or-why-not}{%
\subsubsection{4. Did you determine response rates for the different
questions? Why or why
not?}\label{did-you-determine-response-rates-for-the-different-questions-why-or-why-not}}

Yes. In different analysis we have to consider and determine the
response rate as they provide valuable insight of the accuracy of the
data. The higher response rate in the data means it is more
representative of samples, which is the purpose of the survey. In
different questions, the response rate is the ratio of number of
participants in study to the number of participants that were asked to
join in the survey.

\hypertarget{did-you-consider-weighting-any-of-the-data-if-yes-why}{%
\subsubsection{5. Did you consider weighting any of the data? If yes,
why?}\label{did-you-consider-weighting-any-of-the-data-if-yes-why}}

Yes. For example, from this data we can see in race/ethnicity variable,
the percentage of white is much more than other group, making sample
result biased toward white. This might be the issue of method of
sampling, or they have higher response than other group. What we need to
do is to have the target proportion, divided by the actual proportion of
different race group, to get weight of each different group. In this
way, it can be a rebalance way and can make survey more accurately to
reflect population.

\hypertarget{what-appropriate-tests-of-statistical-significance-did-you-consider-using-and-used-when-evaluating-differences-across-different-sub-populations}{%
\subsubsection{6. What appropriate tests of statistical significance did
you consider using (and used), when evaluating differences across
different
sub-populations?}\label{what-appropriate-tests-of-statistical-significance-did-you-consider-using-and-used-when-evaluating-differences-across-different-sub-populations}}

ANOVA is the method to use evaluating differences across different
sub-population group. The ANOVA's null hypothesis is that there is no
difference in means of different groups, while the alternate hypothesis
is the means are not all equal. There is a package of ANOVA in R to do
this test. And if the p value is great than 0.05, then we will conclude
that there is no difference in means of different groups. Otherwise, we
will come to the conclusion that the means are not all equal.

\end{document}
